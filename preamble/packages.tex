%% Responsible for every package import with its respective desired options
%%
%% (c) Thomas Hesse
%%
%% folder: ./preamble/

%%--------------------
% DEFAULT PACKAGES
\usepackage[
  main=english,%                        % primary language
  ngerman%                              % secondary language
]{babel}                                % language options (primary: english, secondary: german)

\usepackage{graphicx}                      % fix figures (float environment)
\usepackage{tikz}                       % tikz graphics
\usepackage{mathtools} 					% erweiterte Fassung von amsmath
\usepackage{amssymb}   					% erweiterter Zeichensatz
\usepackage{siunitx}   					% Einheiten
\usepackage{amsmath}                    % mathematical formulars
\DeclareMathOperator*{\argmax}{argmax} % thin space, limits underneath in displays

\DeclareMathOperator*{\argmin}{argmin} % thin space, limits underneath in displays

\usepackage{amsthm}                     % definition of custom theorems and definitions
\usepackage[%
  xindy,%                               % use xindy for makeglossaries
  section = section,%                   % use sections for all glossary lists 
  nonumberlist,%                        % no page references in lists
  %sanitize={symbol=false},%             % do not dissect symbols
  shortcuts,%                           % make use of shorthand notation
  acronym,%                             % acronym glossary
  %nopostdot,%                           % remove the point at the end
  nogroupskip,%                         % don't skip any groups abstand bei acronym gruppen
  nomain,%                              % do not generate main glossary
  nowarn%                               % suppress warnings
]{glossaries}[=v4.49]                         % glossary (abbreviations and symbols)
\usepackage{tabularx}                   % table environment used for IAS glossary style
\usepackage{booktabs}   				% Verbesserte Möglichkeiten für Tabellenlayout über horizontale Linien
\usepackage{longtable}                  % table environment used for IAS glossary style
\usepackage{enumitem}                   % less space between vertical items in enumerate
%\usepackage{listings}                   % listings, e.g. for matlab code in the appendix
\usepackage{xcolor}                     % coloring of listings
\usepackage{caption}                    % enumerated captions in figures, etc.
\usepackage{subcaption}                 % subcaptions for subfigures
\usepackage{epstopdf}                   % include eps graphics
% \usepackage{psfrag}                     % modify eps graphics (does not work with matlab eps files)
\usepackage{wrapfig}                    % wraps text around figures
\usepackage[%
  vlined,%                              % design of code blocks
  % boxed,%                              layout of the algorithm
  ruled%
]{algorithm2e}                          % algorithm environment

% \usepackage{algorithm}
% \usepackage{algpseudocode}

% Der folgende Block ist nur bei pdfTeX auf Versionen vor April 2018 notwendig
\usepackage{iftex}
\ifPDFTeX
	\usepackage[utf8]{inputenc}%kompatibilität mit TeX Versionen vor April 2018
\fi

\usepackage{url}
\usepackage{subcaption}
\usepackage{float}

\usepackage[autostyle]{csquotes}		% Anführungszeichen vereinfacht
\usepackage{microtype}


%%--------------------
% DEBUG PACKAGES
\usepackage{comment}
\usepackage{todonotes}

%%--------------------
% USE CUSTOM IAS STYLE
\usepackage{iasThesis}

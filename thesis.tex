%% This is file `DEMO-TUDaThesis.tex' version 2.05 (2019/12/19),
%% it is part of 
%% TUDa-CI -- Corporate Design for TU Darmstadt
%% ----------------------------------------------------------------------------
%%
%%  Copyright (C) 2018--2019 by Marei Peischl <marei@peitex.de>
%%
%% ============================================================================
%% This work may be distributed and/or modified under the
%% conditions of the LaTeX Project Public License, either version 1.3c
%% of this license or (at your option) any later version.
%% The latest version of this license is in
%% http://www.latex-project.org/lppl.txt
%% and version 1.3c or later is part of all distributions of LaTeX
%% version 2008/05/04 or later.
%%
%% This work has the LPPL maintenance status `maintained'.
%%
%% The Current Maintainers of this work are
%%   Marei Peischl <tuda-ci@peitex.de>
%%   Markus Lazanowski <latex@ce.tu-darmstadt.de>
%%
%% The development respository can be found at
%% https://github.com/tudace/tuda_latex_templates
%% Please use the issue tracker for feedback!
%%
%% ============================================================================
%%
% !TeX program = lualatex
%%

\documentclass[
	ruledheaders=section,%Ebene bis zu der die Überschriften mit Linien abgetrennt werden, vgl. DEMO-TUDaPub
	class=report,% Basisdokumentenklasse. Wählt die Korrespondierende KOMA-Script Klasse
	thesis={type=bachelor},% Dokumententyp Thesis, für Dissertationen siehe die Demo-Datei DEMO-TUDaPhd
	accentcolor=1c,% Auswahl der Akzentfarbe
	custommargins=false,%true,% Ränder werden mithilfe von typearea automatisch berechnet
	marginpar=false,% Kopfzeile und Fußzeile erstrecken sich nicht über die Randnotizspalte
	%BCOR=5mm,%Bindekorrektur, falls notwendig
	parskip=half-,%Absatzkennzeichnung durch Abstand vgl. KOMA-Sript
	fontsize=11pt,%Basisschriftgröße laut Corporate Design ist mit 9pt häufig zu klein
	%logofile=example-image, %Falls die Logo Dateien nicht vorliegen
	instbox=false,
	IMRAD=false,
	% twoside,
]{tudapub}


%% add your chapters here
\newcommand{\content}{%
	\chapter{Introduction}
\label{chap:introduction}
This is a citation: \cite{Vaswani2017}


This is a figure: 

\begin{figure}[ht]
    \centering
    \includegraphics[width=.5\textwidth]{figures/AgentEnviornment.png}
    \caption{I am a caption}
    \label{fig:my_label}
\end{figure}%
	\chapter{Background}
\label{chap:background}
This is an algorithm 





\section{Classes of Anomaly detection}
- there are different kinds of approaches to IAD
- look at tree picture
 
- First important distinction is between supervised and unsupervised
-> we focus on unsupervised
-> list problems with supervised approaches and thus advantages of unsupervised ones

- briefly touch on other IAD settings like few shot, along with references

- among unsupervised approaches, there are two more fundamental distinctions
-> reconstruction based vs representation/feature embedding based
-> explain difference with lots of references

- for reconstruction based touch on 2-3 base categories like GANs etc and link fundamental papers for GANs etc
- for representation based important to explain memory bank, teacher student, and distribution map
- explain normalizing flow somehow somewhere in there

- maybe say which algos we chose and what we covered with that


\section{Our own Dataset}
- repeat motivation why we added additional data in mvtec style
- say that we went with loco mvtec flair(maybe give reasons)
- say that we came up with a set of structural and logical anomalies for each category
- list categories(flat connector, angle and special construct)

- 3 sub sections for the three categories

- flat connector
- link the exact one we used(or examples of some)
- give structural anomalies
- give logical anomalies
- for both briefly touch on how we produced them
- show image examples for each

- repeat same for other categories

- also when describing angle:
- touch on how there is a special case with multi perspective detection



\section{metrics}

- show metrics from survey papers
- explain which metrics we used and where the other ones are used
- explain also why we used the ones we used, and what disadvantages of other ones where

- touch on paul bergmann paper for sPRO score, say how it is better than pixel auroc and normal pro score, also explain saturation thresholds

- some math formula for calculating the important metrics






















%{\SetAlgoNoLine%
%\begin{algorithm}[H]

%\end{algorithm}%
%}%
	\chapter{Novel Dataset Class}
\label{chap:datasets}

This chapter introduces the novel dataset category representing a flat connector. Mainly the sections will revolve around the specifics of the flat connector object, the setup used to 
record the data and produce labels, as well as the anomalies that were chosen for this class. Section \ref{sec:flatconnectoranomalies} also reviews the steps that went into producing the specific anomalies 
and how to recreate them.


\section{Flat Connector}
\label{sec:faltconnectordesscription}

As mentioned prior to this chapter, this work will also deal with the introduction of a novel dataset category that is an extension of the MVTecAD LOCO \cite{LOCODentsAndScratchesBergmann2022} dataset. 
Despite simply extending the range of objects covered in the dataset, this also serves a more various investigation of IAD method performance in industrial settings. The motivation behind adhering 
to MVTecAD LOCO standards is multifold. Firstly the setting of this work is already in the logical anomalies context, which greatly facilitates the evaluation of the new class. Additionally, as 
discussed in section \ref{sec:datasets}, there is only a slight technical difference between the MVTecAD LOCO and the classical MVTecAD \cite{MVTEC_Bergmann_2021} dataset, meaning that this novel 
class can easily be evaluated by the vast majority of IAD approaches, as most of them already report their performance on the MVTecAD dataset. This holds true for past approaches but also new ones 
to come.
\newline\newline
The flat connector object was chosen for this since it meets multiple requirements for an adequate object class. It is of metal manufacturing nature, and can comfortably be photographed from 
an overhead perspective. Although the MVTecAD LOCO dataset already contains a screw bag, the property of the plastic bag could potentially shift the focus of the performance from solely the metal 
part performance to the anomaly localization in more difficult conditions. The latter is a common characteristic of the other classes in the set. Finally the flat connector has some 
properties that make it a very favorable manufacturing part to include: It is of a lighter metal, giving us much more freedom to produce anomalies. A steel block for example would be vastly 
harder to meaningfully process into it representing various kinds of anomalies. Moreover the differently sized holes in set configurations that the flat connector possesses, make for a lot of opportunities 
to introduce logical anomalies without the need of arbitrary rules like the pushpin or splicing connector class. For example the latter class may need multiple objects in a certain arrangement per image to then predefine 
rules as logical constraints, whereas the flat connector as such may possess multiple logical violations within only one object without the need of additional artificial rules.
\newline
Further regarding the flat size specifications, we used regulatory flat 
connectors of size $100x35 mm$ which are widely available at hardware stores and at \cite{flatconnectorlink}. The surface of the flat connectors is galvanized 
and displays a CE-label according to DIN EN 14545. Images of normal flat connectors without manipulations can be viewed in appendix figure \ref{fig:flatgoodimages}.




\begin{figure}[ht]
    \captionsetup[subfigure]{justification=centering}
    \centering
    \begin{subfigure}[b]{0.3\textwidth}
        \centering
        \includegraphics[width=0.475\textwidth]{figures/flatconnectordatasetimages/cut_corner_image.png}
        \includegraphics[width=0.475\textwidth]{figures/flatconnectordatasetimages/cut_corner_mask.png}
        \caption{Cut Corner}

    \end{subfigure}
    \hfill
    \begin{subfigure}[b]{0.3\textwidth}
        \centering
        \includegraphics[width=0.475\textwidth]{figures/flatconnectordatasetimages/damaged_edge.png}
        \includegraphics[width=0.475\textwidth]{figures/flatconnectordatasetimages/damaged_edge_mask.png}
        \caption{Damaged Edge}
        \label{fig:damagededgesubfig}
    \end{subfigure}
    \hfill
    \begin{subfigure}[b]{0.3\textwidth}
        \centering
        \includegraphics[width=0.475\textwidth]{figures/flatconnectordatasetimages/scratch.png}
        \includegraphics[width=0.475\textwidth]{figures/flatconnectordatasetimages/scratch_mask.png}
        \caption{Scratch}

    \end{subfigure}
    \\
    \begin{subfigure}[b]{0.3\textwidth}
        \centering
        \includegraphics[width=0.475\textwidth]{figures/flatconnectordatasetimages/missing_big_hole.png}
        \includegraphics[width=0.475\textwidth]{figures/flatconnectordatasetimages/missing_big_hole_mask.png}
        \caption{Missing Big Hole}

    \end{subfigure}
    \hfill
    \begin{subfigure}[b]{0.3\textwidth}
        \centering
        \includegraphics[width=0.475\textwidth]{figures/flatconnectordatasetimages/extra_holes.png}
        \includegraphics[width=0.475\textwidth]{figures/flatconnectordatasetimages/extra_holes_mask.png}
        \caption{Extra Holes}

    \end{subfigure}
    \hfill
    \begin{subfigure}[b]{0.3\textwidth}
        \centering
        \includegraphics[width=0.475\textwidth]{figures/flatconnectordatasetimages/diff_size_holes.png}
        \includegraphics[width=0.475\textwidth]{figures/flatconnectordatasetimages/diff_size_holes_mask.png}
        \caption{Differently Sized Holes}

    \end{subfigure}
    %\captionsetup{justification=centering, skip=-10pt}
    \caption{Example images of anomalous samples from the novel flat connector class. The according class names are written below the subfigures.}
    \label{fig:flatconnectorexampleimages}
\end{figure}






\section{Data Acquisition Setup}
\label{sec:faltconnectordataacquisiton}

The images for the dataset were taken in the university facilities. An exemplary setup for image acquisition can be viewed in figure \ref{fig:setupofdatacollection}. The contraption to capture images was 
done by utilizing a camera mounted to an robotic arm, pointing the camera lens at the table from an overhead perspective. The distance between camera lens and the table surface was approximately 15 cm and the main light source in the 
images was delivered by a built in flash light of the arm. Regarding the background of the object, a black cloth was chosen to be placed on the table, where the object would then be put. For 
an increased variety, multiple flat connectors were bought and filmed. Here each one was manually rotated after each shot, to achieve multiple object orientations and thus a more robust training 
set. Each flat connector was rotated a total of six times per side, resulting in 12 normal images per flat connector object. \newline
The camera used to take pictures produced the images of the flat connectors in a 4k resolution. Due to the large size of the resulting 
images, their size was adjusted by half of the original image size.\newline
To obtain segmentation labels, a free polygon annotation software was used to then label anomalous regions in the images. The annotations were done by hand and by the author of this work. In accord to 
MVTecAD LOCO standards \cite{LOCODentsAndScratchesBergmann2022} each kind of anomaly would be assigned a unique pixel value to differentiate them when working with saturation thresholds 
as mentioned in section \ref{sec:datasets}. Example annotations can be viewed in figure \ref{sec:flatconnectoranomalies}, where several original images are depicted next to the produced segmentation labels. Just like in the 
original MVTecAD LOCO dataset, image labels were not explicitly given since they can be inferred by the anomaly class.


\begin{figure}[htbp]
    \captionsetup[subfigure]{justification=centering}
    \centering
    \begin{subfigure}[b]{0.3\textwidth}
        \centering
        \includegraphics[angle=90, width=\textwidth]{figures/setupimages/setup_empty.JPG}


    \end{subfigure}
    \begin{subfigure}[b]{0.3\textwidth}
        \centering
        \includegraphics[angle=90, width=\textwidth]{figures/setupimages/setup_flachverbinder.JPG}
        %\caption*{Logical Anomalies}

    \end{subfigure}
    \begin{subfigure}[b]{0.3\textwidth}
        \centering
        \includegraphics[angle=90, width=\textwidth]{figures/setupimages/setup_close.JPG}


    \end{subfigure}
    \caption{Manually taken images from the setup used to record the flat connector class. The images depict a wide shot of the robot mount used, 
             a closer image with a flat connector on the table and a close-up of the process.}
    \label{fig:setupofdatacollection}
\end{figure}




\section{Anomalies}
\label{sec:flatconnectoranomalies}

To a certain degree the common MVTecAD LOCO classes all possess a balanced amount of logical and structural anomalies. Therefore the collection of possible anomalies for this class was kept at an even structural 
to logical ratio. The anomalies listed below can be viewed in figure \ref{fig:flatconnectorexampleimages} where exemplary anomalous images are displayed next to their corresponding ground truth mask.

\subsection{Structural Anomalies}
Structural anomalies in this case were comparably simple to think of an execute since they only involved damaging some area of the flat connector. Here we also tried to represent a variety of 
certain characteristics of anomalies among the ones we chose. An example would be to include anomalies with larger areas as well as ones with small ground truths.\newline
First of we produced the structural anomaly of a cut off corner. This was done by simply using a metal saw to remove a corner of the flat connector. This is an anomaly with a comparably large 
surface area and clean edges to predict. Next we again used a metal saw to damage the edges of the flat connector by cutting out pieces of or simply sawing into the edges of the object. One image 
of this anomaly type contained multiple damages spots on multiple edges. Due to the cuts partially being very small, this makes for an anomaly with very small ground truth areas as seen in figure 
\ref{fig:damagededgesubfig}. As a last structural anomaly the saw was used to apply deep scratches on the object. This is mimicking another type of anomaly sometimes found in the classes of splicing connectors and pushpins 
in the original dataset. The common characteristic is that the region is very slim and long, making for better comparisons.

\subsection{Logical Anomalies}
The logical anomalies comprise the more interesting part of the anomaly types, as they are the focus of the MVTecAD LCOO dataset. The first logical anomaly in this class is the case in which a hole 
is missing in the flat connector hole pattern. This anomaly had to be improvised, since it is difficult to obtain a properly manufactured flat connector that was faulty produced. To still include 
this type of anomaly, the solution was to first fill the hole with moldable material and then spray paint over it to hide color differences. For realism the spray paint color was the same as the 
object surface and the whole object was spray painted to avoid color artifacts. Partially this anomaly was one of the larger and more clean cut ones since the holes mostly chosen to be filled were one of the two 
bigger ones on the object. For increased variety there were also anomalous cases added where a small hole would be concealed. As a second anomaly the case of an extra hole was realized. To produce such an anomaly the facilities offered a drill, which was used to drill another small hole into 
the object at an unusual location. The size of the additional hole was the same as the smaller ones on the object. To also test capabilities of detecting multiple anomalies in a single image, this anomaly 
was put together with an differently sized hole. The final anomaly to be introduced was a resized hole at a correct 
location. This would mean that the anomaly does not consist of the hole location but rather the size difference between this holes and the other ones in similar places. For this type of anomaly it 
is admissible to predict the whole hole as an anomalous region.  %vor der abgabe maybe die labels dafür noch anpassen dass es ringe sind
Here a drill tip was used with a slightly higher radius that the original hole to enlarge the diameter.


\subsection{Saturation Thresholds}

The saturation scores for the anomalies, as discussed in section \ref{sec:datasets} were put at the amount of pixels in the anomalous region for all above listed anomalies respectively.
This was because the nature of presented anomalies of this categories calls for a full segmentation of the respective anomaly for a perfect result. Unlike in the pushpin example given in that section 
there are no cases that warrant multiple possible placements of anomalies.






%
	\chapter{Method}
\label{chap:method}
This is an table:
\input{IAS-thesis-MSRL/algotable/table}%
	\chapter{Experimental Setup}
\label{chap:experimentsetup}

All models trainings and result reproductions have been conducted on the IAD cluster student partition. The GPU in use for all nodes used by that partition (ist die formulierung gut?) is and 
RTX 2080Ti with 11GB of memory, and the CPU is an AMD Ryzen 9 16-Core processor. The cluster overview (verlinken) serves to provide further detail for additional questions. As for software, 
pytorch 2.1.2 was utilized to implement the ensemble model. The specifications of other libraries, as well as the specifications for the MVTecAD LOCO experiment are documented in the 
environment files in the project code.

The dataset category was recorded with a (felix nach infos fragen) camera, which was stationary mounted over a surface with a black cloth on it at a height of (felix fragen) centimeters. 
The lighting of the photoshoot was provided by a softbox(???) (felix für mehr infos fragen).





%
	\chapter{Experimental Results}
\label{chap:experiments}



- analysis on how methods worked on own dataset individually
-> if poor performance error analysis and also address different subclasses

- analysis of how ensemble model worked and if it improved performance



\section{SOTA Methods Performance on classical LOCO Dataset}
\label{sec:locoresultssota}
In this section we review the performance of prior introduced anomaly detection methods. All experiments were performed with the same 
experimental setup as explained in section (referenz of experimental setup section), the conditions explained in section (referenz von methods section über loco) 
and on the mvtec LOCO dataset \cite{LOCODentsAndScratchesBergmann2022}. 
The results of inference on the test set can be seen in table x (tabelle mit ergebnissen). As it can be seen, all models scored a significantly 
lower result on the MVTecAD LOCO dataset than on the normal MVTecAD one(exemplary scores seen in table xy(table mit normalen mvtec scores)). 
A lower performance is generally to be expected, since firstly logial anomalies are regarded as a more difficult problem than structual 
ones and secondly the average SOTA performances as seen in table x(tabelle mit ergebnissen) is already closing in on an AUROC of of 1. 
(den satz rechts von hier müsste man maybe rausmachen oder umschreiben)Therefore there is not much room for further improvement in similar settings, and a worse performance still aknowledgeable as very good. 
Yet there is an drop in cross-model average AUROC of approcimately (durchschnitts drop ausrechnen), which is a remarkable(synonym) difference. 
Most other metrics, namely (metrics names), also declined with an respective average of (respective averages). As explained in section 
(referenz zu metrics section von background), the sPRO (or rather AU-sPRO) was a score introduced in \cite{LOCODentsAndScratchesBergmann2022} to gain an 
advanced insight on the quality of segmentations. This means that all approaches who either were published before or did not include this 
paper in their research likely did not include this metric, which holds true for the approahces used for this experiment. Therefore no comparison 
in sPRO/AU-sPRO can be shown(vllt einfach spro auch für allte ansätze implementieren?? dann kann ich den satz ändern). Comparing the sPRO 
scores of the SOTA methods in this experiment with the ones from compared to GCAD \cite{LOCODentsAndScratchesBergmann2022} shows asignificantly 
(abchecken ob wirklich) worse performance.
Among the different models, the highest scoring one was PatchCore \cite{patchCore2022}. It scored an average (metrics einfügen) feature embedding based approaches like  
achieved the highest scoring

Interpretation of results hier, weiß nicht in welche section das eigentlich muss:



\section{Ensemble Performance}
Notizen für diese section:
- hier soll reportet werden wie das ensemble sich geschlagen hat
- dazu brauche ich:
-> metriken(AUROC sPRO und vllt pixel auroc) von dem ensemble auf flat connector + mvtec loco
-> beispielhafte segmentierungen
-> plots von loss und auroc über training

- drauf eingehen wo sich das ensemble wie gut geschlagen hat
-> vergleich mit patchcore und simplenet wichtig, gerne auch mit DRAEM vergleichen als reconstruction representativer algo
-> sagen bei welchen klassen es gut und nicht so gut geklappt hat, vergleichen mit ergebnissen aus LOCO studie oben drüber(vllt in conclusion?)
-> mehr images in appendix anbieten





%
	\chapter{Conclusion and Future work}
\label{chap:conclusion}

\section{Ensemble Network}
\label{sec:ensembleconclusion}

\section{SOTA performance}
\label{sec:sotaperformance}

\section{Flat connector}
\label{sec:flatconnector}

\section{Outlook}
\label{sec:outlook}%
}



%% Responsible for every package import with its respective desired options
%%
%% (c) Thomas Hesse
%%
%% folder: ./preamble/

%%--------------------
% DEFAULT PACKAGES
\usepackage[
  main=english,%                        % primary language
  ngerman%                              % secondary language
]{babel}                                % language options (primary: english, secondary: german)

\usepackage{graphicx}                      % fix figures (float environment)
\usepackage{tikz}                       % tikz graphics
\usepackage{mathtools} 					% erweiterte Fassung von amsmath
\usepackage{amssymb}   					% erweiterter Zeichensatz
\usepackage{siunitx}   					% Einheiten
\usepackage{amsmath}                    % mathematical formulars
\DeclareMathOperator*{\argmax}{argmax} % thin space, limits underneath in displays

\DeclareMathOperator*{\argmin}{argmin} % thin space, limits underneath in displays

\usepackage{amsthm}                     % definition of custom theorems and definitions
\usepackage[%
  xindy,%                               % use xindy for makeglossaries
  section = section,%                   % use sections for all glossary lists 
  nonumberlist,%                        % no page references in lists
  %sanitize={symbol=false},%             % do not dissect symbols
  shortcuts,%                           % make use of shorthand notation
  acronym,%                             % acronym glossary
  %nopostdot,%                           % remove the point at the end
  nogroupskip,%                         % don't skip any groups abstand bei acronym gruppen
  nomain,%                              % do not generate main glossary
  nowarn%                               % suppress warnings
]{glossaries}[=v4.49]                         % glossary (abbreviations and symbols)
\usepackage{tabularx}                   % table environment used for IAS glossary style
\usepackage{booktabs}   				% Verbesserte Möglichkeiten für Tabellenlayout über horizontale Linien
\usepackage{longtable}                  % table environment used for IAS glossary style
\usepackage{enumitem}                   % less space between vertical items in enumerate
%\usepackage{listings}                   % listings, e.g. for matlab code in the appendix
\usepackage{xcolor}                     % coloring of listings
\usepackage{caption}                    % enumerated captions in figures, etc.
\usepackage{subcaption}                 % subcaptions for subfigures
\usepackage{epstopdf}                   % include eps graphics
% \usepackage{psfrag}                     % modify eps graphics (does not work with matlab eps files)
\usepackage{wrapfig}                    % wraps text around figures
\usepackage[%
  vlined,%                              % design of code blocks
  % boxed,%                              layout of the algorithm
  ruled%
]{algorithm2e}                          % algorithm environment

% \usepackage{algorithm}
% \usepackage{algpseudocode}

% Der folgende Block ist nur bei pdfTeX auf Versionen vor April 2018 notwendig
\usepackage{iftex}
\ifPDFTeX
	\usepackage[utf8]{inputenc}%kompatibilität mit TeX Versionen vor April 2018
\fi



\usepackage[autostyle]{csquotes}		% Anführungszeichen vereinfacht
\usepackage{microtype}


%%--------------------
% DEBUG PACKAGES
\usepackage{comment}
\usepackage{todonotes}

%%--------------------
% USE CUSTOM IAS STYLE
\usepackage{iasThesis}

%% Gloassaries (acronyms, symbols and functions)
%% Glossary entries will be created at the end of the file!
%%
%% (c) Thomas Hesse
%%
%% folder: preamble/

%% acronyms:
%\newacronym{iid}{i.i.d.}{independently and identically distributed}
\newacronym{RL}{RL}{Reinforcement Learning}
\newacronym{ML}{ML}{Machine Learning}

\newacronym{sac}{SAC}{Soft Actor Critic}
\newacronym{PPO}{PPO}{Proximal Policy Optimization}
\newacronym{TRPO}{TRPO}{Trust Region Policy Optimization}
\newacronym{DQN}{DQN}{Deep Q Network}
\newacronym{DDPG}{DDPG}{Deep Deterministic Policy Gradient}

\newacronym{IAD}{IAD}{Image Anomaly Detection}




%% SYMBOLS
\newglossary[syg]{symbol}{sys}{syo}{List of Symbols}
% \newglossaryentry{theta}{
%   sort        = {parameters theta},
%   name        = {\ensuremath{\cvec{\theta}}},
%   description = {vector of parameters from a probability distribution},
%   type        = symbol
% }

\newglossaryentry{pi}{
  % sort        = {parameters theta},
  name        = {\ensuremath{\pi(a|s_t)}},
  description = {Policy},
  type        = symbol
}

\newglossaryentry{S}{
  % sort        = {parameters theta},
  name        = {\ensuremath{S}},
  description = {continuous state space},
  type        = symbol
}

\newglossaryentry{A}{
  % sort        = {parameters theta},
  name        = {\ensuremath{A}},
  description = {continuous action space},
  type        = symbol
}

\newglossaryentry{entropy}{
  % sort        = {parameters theta},
  name        = {\ensuremath{\mathcal{H}(\cdot)}},
  description = {entropy},
  type        = symbol
}

%% CREATE GLOSSARY
%\makeglossaries
%\makenoidxglossaries
%%%%

\input{preamble/tikzGraphs}


% Der folgende Block ist nur bei pdfTeX auf Versionen vor April 2018 notwendig
\usepackage{iftex}
\ifPDFTeX
	\usepackage[utf8]{inputenc}%kompatibilität mit TeX Versionen vor April 2018
\fi


\begin{document}

	\pagenumbering{gobble}                    % suppress page numbering
	  
	\Metadata{
		title=Ensemble Methods for Visual Anomaly Detection in Manufacturing Settings,
		author=Marc Saghir
	}
	
	\title{Ensemble Methods for Visual Anomaly Detection in Manufacturing Settings}
	\subtitle{Ensemble-Methoden für die Erkennung von visuellen Anomalien in Produktionsumgebungen}
	\author[M. Saghir]{Marc Saghir} %optionales Argument ist die Signatur,
	\birthplace{Darmstadt}%Geburtsort, bei Dissertationen zwingend notwendig
	
	\reviewer{M.Sc. Suman Pal \and Prof. Jan Peters, Ph.D.}

	%Diese Felder werden untereinander auf der Titelseite platziert.
	%\department ist eine notwendige Angabe, siehe auch dem Abschnitt `Abweichunsg von den Vorgaben für die Titelseite'
	\department{inf} % Das Kürzel wird automatisch ersetzt und als Studienfach gewählt, siehe Liste der Kürzel im Dokument.
	%\institute{Intelligent Autonomous Systems}
	% \group{iROSA}
	\addTitleBoxLogo*{\includegraphics[width=0.75\linewidth]{img/iasLogo.jpeg}}
	
	
	\submissiondate{\today}
	\examdate{\today}
	
	%\tuprints{urn=1234,printid=12345}
	%\dedication{Für alle, die \TeX{} nutzen.}
	
	\maketitle



	% Use \areaset to set up a new text area (this does not change the layout)
	\areaset{\textwidth}{\textheight}

	% Now you can access the calculated margins
	\typeout{Left margin: \the\oddsidemargin}
	\typeout{Right margin: \the\evensidemargin}


	\affidavit
	
	%%
	\begin{abstract}

Image anomaly detection (IAD) has been very popular in recent years. This circumstance can largely be attributed to its application as quality control in manufacturing 
settings, such as industrial ones. It has had many breakthroughs already. State-of-the-art classifiers achieve a classification and localization performance of up to $99.6$ percent 
on various datasets. In IAD, there is a distinction between two types of anomalies: structural and logical. Structural anomalies denote material damage to the class object, like 
a scratch, while logical ones 
comprise anomalies that violate a set of rules and are not necessarily constrained to one region. An example of a logical anomaly would be a missing object or one pointing in a 
direction not permitted by the rule set of the class. \newline 
Current research in anomaly detection has predominantly focused on evaluating classifiers solely on structural anomalies. Moreover, popular dataset images in this area are often 
in a clinical setting. This approach is problematic as it neglects the detection of logical anomalies that are often critical in real-world applications. This results in 
IAD classifiers, with initially robust and high performance, to significantly drop in detection and localization quality. This drop in performance may range from six to 
twenty-nine percent, which is significant considering the previously mentioned high performance.\newline
To enhance robustness, a feature-level ensemble pipeline is proposed. It integrates diverse feature representations and existing anomaly detection architectures. 
By utilizing feature representations from multiple extractors 
and projecting them into a unified space, the ensemble aims to utilize increased information, thereby improving precision and robustness in anomaly localization. Additionally, 
this work conducts a survey of representative IAD approaches on a novel dataset that includes logical anomalies, offering an extensive overview of current methods and their 
potential for logical anomaly localization. Furthermore, a novel dataset category is introduced, extending the primary dataset used in this study. It features both logical and 
structural anomalies.\newline
The experiments confirm that classifiers' performance significantly decreases when structural and logical anomalies are considered, validating initial 
assumptions. Specific characteristics and cross-classifier comparisons are discussed in greater detail. However, the ensemble approach did not enhance performance or robustness. 
The results were significantly inferior to those of individual ensemble members. Further investigations have shown that the root cause for this lies within the method of 
feature combination.
This outcome highlights the need for further research into feature-level ensembling methods, especially in this context. 
Despite not presenting extraordinary difficulty, the new dataset category provided insights into classifier performance and served as a valid extension of the 
logical dataset. Future studies regarding this topic may utilize this dataset class as the base for a new set of logical anomalies to further expand the variety of logical 
datasets for IAD.

\end{abstract}
%
%
\selectlanguage{ngerman} % select german language
\begin{abstract}%[2]
%	Hier können Sie Ihre deutsche Zusammenfassung schreiben. %

Die Erkennung von Bildanomalien ist in den letzten Jahren sehr beliebt geworden. Dieser Umstand ist vor allem auf die otentielle Anwendung als Qualitätskontrolle 
in industriellen Fabriken zurückzuführen. Es hat bereits viele Durchbrüche in den letzter Zeit gegeben. Aktuelle Modelle erreichen eine Klassifizierungs- und 
Lokalisierungsleistung von bis zu $99,6$ Prozent auf verschiedenen Datensätzen. In dem Gebiet wird zwischen zwei Arten von Anomalien unterschieden: strukturelle und 
logische. Strukturelle Anomalien bezeichnen materielle Schäden am Objekten, wie ein Kratzer, während logische Anomalien 
solche umfassen, die vordefinierte Regeln verletzen oder nicht potentiell mehrere Regionen überspannen. Ein Beispiel für eine logische Anomalie wäre ein fehlendes 
Objekt oder eines das in eine Richtung zeigt, die nach den Klassenregeln nicht zulässig ist. \newline
Die derzeitige Forschung auf dem Gebiet der Anomalieerkennung konzentriert sich hauptsächlich auf die Bewertung von Modellen, die ausschließlich auf strukturelle 
Anomalien basieren. Darüber hinaus sind die Inhalte der gängigen Datensätze in diesem Bereich oft sehr künstlich hergestellt worden. Dies ist problematisch, da es die 
Erkennung von logischen Anomalien vernachlässigt, die in realen Anwendungen oft einen relevanten Mehrwert bringen. Dies führt dazu, dass
Klassifikatoren, die anfangs robuste und hohe Leistungen erbringen, erheblich an Erkennungs- und Lokalisierungsqualität verlieren. Dieser Leistungsabfall kann zwischen 
sechs und neunundzwanzig Prozent betragen, was in Anbetracht der zuvor erwähnten hohen Leistung beträchtlich ist.\newline
Um die Robustheit dieser Modelle zu verbessern, wird hier eine Ensemble-Pipeline auf Merkmalsebene vorgestellt. Sie integriert verschiedene Merkmalsrepräsentationen und 
bestehende Architekturen zur Erkennung von Anomalien. 
Durch die Verwendung von Merkmalsrepräsentationen aus mehreren Extraktoren 
und deren Projektion in eine vergleichbare Repräsentation, zielt das Ensemble darauf ab, mehr Informationen einzubeziehen und dadurch die Präzision und Robustheit bei der 
Lokalisierung von Anomalien zu verbessern. Zusätzlich verschafft diese Arbeit einen Überblick über repräsentative Anomalieerkennungsansätze auf einem neueren Datensatz, welcher 
logische Anomalien enthält, und bietet eine Analyse bezüglich deren 
Potenzial für die Lokalisierung logischer Anomalien. Darüber hinaus wird eine neue Datensatzkategorie eingeführt, die den in dieser Studie verwendeten Datensatz erweitert. 
Sie umfasst sowohl logische als auch strukturelle Anomalien. \newline

Experimente bestätigen, dass die Leistung der Modelle deutlich abnimmt, wenn sowohl strukturelle als auch logische Anomalien berücksichtigt werden, was die 
ursprünglichen Annahmen bestätigt. Spezifische Merkmale und Vergleiche zwischen den Klassifikatoren werden ebenfalls ausführlicher erörtert. Der Ensemble-Ansatz hat 
jedoch weder die Leistung noch die Robustheit verbessert. 
Die Segmentierungen waren deutlich schlechter als die der einzelnen Ensemblemitglieder. Weitere Untersuchungen haben gezeigt, dass die Ursache hierfür in der Methode der 
Merkmalsvereinigung liegt.
Dieses Ergebnis unterstreicht die Notwendigkeit weiterer Forschung zu Methoden der Merkmalsvereinigung, insbesondere in diesem Zusammenhang. 
Obwohl die neue Datensatzkategorie keine außergewöhnlichen Schwierigkeiten für dir Modelle aufwies, lieferte sie sinnvollen Einblicke in die Klassifikatorleistung und diente 
als gelungene Erweiterung des 
logischen Datensatzes. Künftige Studien zu diesem Thema könnten diese Klasse als Grundlage für einen neuen Satz logischer Anomalien verwenden, um die Vielfalt der logischen 
Datensätze für IAD zu erweitern.



\end{abstract}
\selectlanguage{english} % reset to english language

	%%

	
	\tableofcontents
	
	\chapter*{Figures~and~Tables}
	\begingroup                               % LOF and LOT on the same page
	\let\clearpage\relax
	\listoffigures\listoftables
	\endgroup
	
	
	%\chapter*{Abbreviations,~Symbols and Operators}
	%\glsaddall
    
    %\printnoidxglossary
    
	%\printnoidxglossary[type=acronym,title=List of Abbreviations, style=iasThesisGeneral]
	%\printnoidxglossary[type=symbol,style=iasThesisGeneral]
	%\printnoidxglossary[type=operator,style=iasThesisOperators]
	
	\pagenumbering{arabic}                    % start normal page numbering
	
	\content                                  % insert the main content
	
	\cleardoublepage
	\bibliography{literature/lit}             % references (for options see usepackage section)
	
%	% this causes errors for some reason	
%	\nocite{*}                                % display everything, even if not cited (should be commented in the end!)
	
	\cleardoublepage
	\appendix\appendix
\chapter{Appendix}


%\begin{figure}
    \centering
    \begin{subfigure}[b]{0.44\textwidth}
        \centering
        \includegraphics[width=\textwidth]{figures/represbased_viz.png}
        \caption{Representation Based Approaches}
        \label{subfig:repbased}
    \end{subfigure}
    \hfill
    \begin{subfigure}[b]{0.44\textwidth}
        \centering
        \includegraphics[width=\textwidth]{figures/rec_based.png}
        \caption{Reconstruction Based Approaches}
        \label{subfig:recbased}
    \end{subfigure}
    \caption{Visualizations for the general process of representation and reconstruction based IAD methods.}
    \label{fig:vizofrecrepbased}
\end{figure}

\begin{figure}[htbp]
    \captionsetup[subfigure]{justification=centering}
    \centering
    \begin{subfigure}[b]{0.25\textwidth} % Decreased width to add space
        \centering
        \includegraphics[width=\textwidth]{figures/flatconnectorgoodimages/278.png}
    \end{subfigure}
    \hspace{0.05\textwidth} % Add space between subfigures
    \begin{subfigure}[b]{0.25\textwidth} % Decreased width to add space
        \centering
        \includegraphics[width=\textwidth]{figures/flatconnectorgoodimages/142.png}
    \end{subfigure}
    \hspace{0.05\textwidth} % Add space between subfigures
    \begin{subfigure}[b]{0.25\textwidth} % Decreased width to add space
        \centering
        \includegraphics[width=\textwidth]{figures/flatconnectorgoodimages/255.png}
    \end{subfigure}
    \caption{Example images of normal flat connectors.}
    \label{fig:flatgoodimages}
\end{figure}

\begin{figure}[H]
    \captionsetup[subfigure]{justification=centering}
    \centering
    \begin{subfigure}[b]{\textwidth}
        \centering
        \includegraphics[width=0.45\textwidth]{figures/appendix/appendix_main_ensemble/BB/image_prediction_128.png}
        \includegraphics[width=0.45\textwidth]{figures/appendix/appendix_main_ensemble/BB/image_prediction_272.png}

    \end{subfigure}
    \begin{subfigure}[b]{\textwidth}
        \centering
        \includegraphics[width=0.45\textwidth]{figures/appendix/appendix_main_ensemble/JB/image_prediction_268.png}
        \includegraphics[width=0.45\textwidth]{figures/appendix/appendix_main_ensemble/JB/image_prediction_306.png}

    \end{subfigure}
    \begin{subfigure}[b]{\textwidth}
        \centering
        \includegraphics[width=0.45\textwidth]{figures/appendix/appendix_main_ensemble/PP/image_prediction_199.png}
        \includegraphics[width=0.45\textwidth]{figures/appendix/appendix_main_ensemble/PP/image_prediction_308.png}

    \end{subfigure}
    \begin{subfigure}[b]{\textwidth}
        \centering
        \includegraphics[width=0.45\textwidth]{figures/appendix/appendix_main_ensemble/SB/image_prediction_140.png}
        \includegraphics[width=0.45\textwidth]{figures/appendix/appendix_main_ensemble/SB/image_prediction_307.png}

    \end{subfigure}
    \begin{subfigure}[b]{\textwidth}
        \centering
        \includegraphics[width=0.45\textwidth]{figures/appendix/appendix_main_ensemble/SC/image_prediction_167.png}
        \includegraphics[width=0.45\textwidth]{figures/appendix/appendix_main_ensemble/SC/image_prediction_261.png}

    \end{subfigure}
    
    \caption{Example segmentations of the stacking ensemble approach on the MVTecAD LOCO \cite{LOCODentsAndScratchesBergmann2022} dataset.}
    \label{fig:appendixEnsemble}
\end{figure}

\begin{figure}[H]
    \captionsetup[subfigure]{justification=centering}
    \centering
    \begin{subfigure}[b]{\textwidth}
        \centering
        \includegraphics[width=0.45\textwidth]{figures/appendix/appendix_patchcore/BB/breakfast_box_test_logical_anomalies_034.png}
        \hfill
        \includegraphics[width=0.45\textwidth]{figures/appendix/appendix_patchcore/BB/breakfast_box_test_structural_anomalies_024.png}

    \end{subfigure}
    \begin{subfigure}[b]{\textwidth}
        \centering
        \includegraphics[width=0.45\textwidth]{figures/appendix/appendix_patchcore/JB/juice_bottle_test_logical_anomalies_090.png}
        \hfill
        \includegraphics[width=0.45\textwidth]{figures/appendix/appendix_patchcore/JB/juice_bottle_test_structural_anomalies_037.png}

    \end{subfigure}
    \begin{subfigure}[b]{\textwidth}
        \centering
        \includegraphics[width=0.45\textwidth]{figures/appendix/appendix_patchcore/PP/pushpins_test_logical_anomalies_025.png}
        \hfill
        \includegraphics[width=0.45\textwidth]{figures/appendix/appendix_patchcore/PP/pushpins_test_structural_anomalies_064.png}

    \end{subfigure}
    \begin{subfigure}[b]{\textwidth}
        \centering
        \includegraphics[width=0.45\textwidth]{figures/appendix/appendix_patchcore/SB/screw_bag_test_logical_anomalies_001.png}
        \hfill
        \includegraphics[width=0.45\textwidth]{figures/appendix/appendix_patchcore/SB/screw_bag_test_logical_anomalies_097.png}

    \end{subfigure}
    \begin{subfigure}[b]{\textwidth}
        \centering
        \includegraphics[width=0.45\textwidth]{figures/appendix/appendix_patchcore/SC/splicing_connectors_test_logical_anomalies_003.png}
        \hfill
        \includegraphics[width=0.45\textwidth]{figures/appendix/appendix_patchcore/SC/splicing_connectors_test_structural_anomalies_058.png}

    \end{subfigure}
    
    \caption{Example segmentations of the PatchCore \cite{patchCore2022} approach on the MVTecAD LOCO \cite{LOCODentsAndScratchesBergmann2022} dataset.}
    \label{fig:appendixpatchcore}
\end{figure}

\begin{figure}[H]
    \captionsetup[subfigure]{justification=centering}
    \centering
    \begin{subfigure}[b]{\textwidth}
        \centering
        \includegraphics[width=0.45\textwidth]{figures/appendix/appendix_simplenet/BB/image_prediction_131.png}
        \includegraphics[width=0.45\textwidth]{figures/appendix/appendix_simplenet/BB/image_prediction_214.png}

    \end{subfigure}
    \begin{subfigure}[b]{\textwidth}
        \centering
        \includegraphics[width=0.45\textwidth]{figures/appendix/appendix_simplenet/JB/image_prediction_165.png}
        \includegraphics[width=0.45\textwidth]{figures/appendix/appendix_simplenet/JB/image_prediction_238.png}

    \end{subfigure}
    \begin{subfigure}[b]{\textwidth}
        \centering
        \includegraphics[width=0.45\textwidth]{figures/appendix/appendix_simplenet/PP/image_prediction_160.png}
        \includegraphics[width=0.45\textwidth]{figures/appendix/appendix_simplenet/PP/image_prediction_195.png}

    \end{subfigure}
    \begin{subfigure}[b]{\textwidth}
        \centering
        \includegraphics[width=0.45\textwidth]{figures/appendix/appendix_simplenet/SB/image_prediction_150.png}
        \includegraphics[width=0.45\textwidth]{figures/appendix/appendix_simplenet/SB/image_prediction_311.png}

    \end{subfigure}
    \begin{subfigure}[b]{\textwidth}
        \centering
        \includegraphics[width=0.45\textwidth]{figures/appendix/appendix_simplenet/SC/image_prediction_140.png}
        \includegraphics[width=0.45\textwidth]{figures/appendix/appendix_simplenet/SC/image_prediction_298.png}

    \end{subfigure}
    
    \caption{Example segmentations of the SimpleNet \cite{liu2023simplenet} approach on the MVTecAD LOCO \cite{LOCODentsAndScratchesBergmann2022} dataset.}
    \label{fig:appendixpatchcore}
\end{figure}

\begin{figure}[H]
    \centering
    \begin{subfigure}[b]{\textwidth}
        \centering
        \begin{minipage}{0.45\textwidth}
            \centering
            \includegraphics[width=0.3\textwidth]{figures/appendix/appendix_RevDist/BB/010.png}
            \includegraphics[width=0.3\textwidth]{figures/appendix/appendix_RevDist/BB/010_mask.png}
            \includegraphics[width=0.3\textwidth]{figures/appendix/appendix_RevDist/BB/010segment.png}
        \end{minipage}
        \begin{minipage}{0.45\textwidth}
            \centering
            \includegraphics[width=0.3\textwidth]{figures/appendix/appendix_RevDist/BB/038.png}
            \includegraphics[width=0.3\textwidth]{figures/appendix/appendix_RevDist/BB/038_mask.png}
            \includegraphics[width=0.3\textwidth]{figures/appendix/appendix_RevDist/BB/038segment.png}
        \end{minipage}
    \end{subfigure}
    \hfill
    \begin{subfigure}[b]{\textwidth}
        \centering
        \begin{minipage}{0.45\textwidth}
            \centering
            \includegraphics[width=0.3\textwidth]{figures/appendix/appendix_RevDist/JB/007.png}
            \includegraphics[width=0.3\textwidth]{figures/appendix/appendix_RevDist/JB/007_mask.png}
            \includegraphics[width=0.3\textwidth]{figures/appendix/appendix_RevDist/JB/007_m.png}
        \end{minipage}
        \begin{minipage}{0.45\textwidth}
            \centering
            \includegraphics[width=0.3\textwidth]{figures/appendix/appendix_RevDist/JB/013.png}
            \includegraphics[width=0.3\textwidth]{figures/appendix/appendix_RevDist/JB/013_mask.png}
            \includegraphics[width=0.3\textwidth]{figures/appendix/appendix_RevDist/JB/013_m.png}
        \end{minipage}
    \end{subfigure}
    \hfill
    \begin{subfigure}[b]{\textwidth}
        \centering
        \begin{minipage}{0.45\textwidth}
            \centering
            \includegraphics[width=0.3\textwidth]{figures/appendix/appendix_RevDist/PP/003.png}
            \includegraphics[width=0.3\textwidth]{figures/appendix/appendix_RevDist/PP/003_mask.png}
            \includegraphics[width=0.3\textwidth]{figures/appendix/appendix_RevDist/PP/003_m.png}
        \end{minipage}
        \begin{minipage}{0.45\textwidth}
            \centering
            \includegraphics[width=0.3\textwidth]{figures/appendix/appendix_RevDist/PP/008.png}
            \includegraphics[width=0.3\textwidth]{figures/appendix/appendix_RevDist/PP/008_mask.png}
            \includegraphics[width=0.3\textwidth]{figures/appendix/appendix_RevDist/PP/008_m.png}
        \end{minipage}
    \end{subfigure}
    \hfill
    \begin{subfigure}[b]{\textwidth}
        \centering
        \begin{minipage}{0.45\textwidth}
            \centering
            \includegraphics[width=0.3\textwidth]{figures/appendix/appendix_RevDist/SB/002.png}
            \includegraphics[width=0.3\textwidth]{figures/appendix/appendix_RevDist/SB/002_mask.png}
            \includegraphics[width=0.3\textwidth]{figures/appendix/appendix_RevDist/SB/002m.png}
        \end{minipage}
        \begin{minipage}{0.45\textwidth}
            \centering
            \includegraphics[width=0.3\textwidth]{figures/appendix/appendix_RevDist/SB/008.png}
            \includegraphics[width=0.3\textwidth]{figures/appendix/appendix_RevDist/SB/008_mask.png}
            \includegraphics[width=0.3\textwidth]{figures/appendix/appendix_RevDist/SB/008m.png}
        \end{minipage}
    \end{subfigure}
    \hfill
    \begin{subfigure}[b]{\textwidth}
        \centering
        \begin{minipage}{0.45\textwidth}
            \centering
            \includegraphics[width=0.3\textwidth]{figures/appendix/appendix_RevDist/SC/021.png}
            \includegraphics[width=0.3\textwidth]{figures/appendix/appendix_RevDist/SC/021_mask.png}
            \includegraphics[width=0.3\textwidth]{figures/appendix/appendix_RevDist/SC/021m.png}
        \end{minipage}
        \begin{minipage}{0.45\textwidth}
            \centering
            \includegraphics[width=0.3\textwidth]{figures/appendix/appendix_RevDist/SC/045.png}
            \includegraphics[width=0.3\textwidth]{figures/appendix/appendix_RevDist/SC/045_mask.png}
            \includegraphics[width=0.3\textwidth]{figures/appendix/appendix_RevDist/SC/045m.png}
        \end{minipage}
    \end{subfigure}


    \caption{Example segmentations of the Reverse Distillation \cite{revdist2023} approach on the MVTecAD LOCO \cite{LOCODentsAndScratchesBergmann2022} dataset.}
    \label{fig:appendixRevDist}
\end{figure}

\begin{figure}[H]
    \centering
    \begin{subfigure}[b]{\textwidth}
        \centering
        \begin{minipage}{0.45\textwidth}
            \centering
            \includegraphics[width=\textwidth]{figures/appendix/appendix_DRAEM/BB/129.png}
            \includegraphics[width=\textwidth]{figures/appendix/appendix_DRAEM/BB/129_m.png}
            \includegraphics[width=\textwidth]{figures/appendix/appendix_DRAEM/BB/129_mask.png}
        \end{minipage}
        \begin{minipage}{0.45\textwidth}
            \centering
            \includegraphics[width=\textwidth]{figures/appendix/appendix_DRAEM/BB/226.png}
            \includegraphics[width=\textwidth]{figures/appendix/appendix_DRAEM/BB/226_m.png}
            \includegraphics[width=\textwidth]{figures/appendix/appendix_DRAEM/BB/226_mask.png}
        \end{minipage}
    \end{subfigure}
    
    \begin{subfigure}[b]{\textwidth}
        \centering
        \begin{minipage}{0.45\textwidth}
            \centering
            \includegraphics[width=\textwidth]{figures/appendix/appendix_DRAEM/JB/139.png}
            \includegraphics[width=\textwidth]{figures/appendix/appendix_DRAEM/JB/139_m.png}
            \includegraphics[width=\textwidth]{figures/appendix/appendix_DRAEM/JB/139_mask.png}
        \end{minipage}
        \begin{minipage}{0.45\textwidth}
            \centering
            \includegraphics[width=\textwidth]{figures/appendix/appendix_DRAEM/JB/280.png}
            \includegraphics[width=\textwidth]{figures/appendix/appendix_DRAEM/JB/280_m.png}
            \includegraphics[width=\textwidth]{figures/appendix/appendix_DRAEM/JB/280_mask.png}
        \end{minipage}
    \end{subfigure}

    \begin{subfigure}[b]{\textwidth}
        \centering
        \begin{minipage}{0.45\textwidth}
            \centering
            \includegraphics[width=\textwidth]{figures/appendix/appendix_DRAEM/PP/l010.png}
            \includegraphics[width=\textwidth]{figures/appendix/appendix_DRAEM/PP/10_m.png}
            \includegraphics[width=\textwidth]{figures/appendix/appendix_DRAEM/PP/logical10.png}
        \end{minipage}
        \begin{minipage}{0.45\textwidth}
            \centering
            \includegraphics[width=\textwidth]{figures/appendix/appendix_DRAEM/PP/s010.png}
            \includegraphics[width=\textwidth]{figures/appendix/appendix_DRAEM/PP/s10m.png}
            \includegraphics[width=\textwidth]{figures/appendix/appendix_DRAEM/PP/structural10_mask.png}
        \end{minipage}
    \end{subfigure}

    \begin{subfigure}[b]{\textwidth}
        \centering
        \begin{minipage}{0.45\textwidth}
            \centering
            \includegraphics[width=\textwidth]{figures/appendix/appendix_DRAEM/SB/144.png}
            \includegraphics[width=\textwidth]{figures/appendix/appendix_DRAEM/SB/144m.png}
            \includegraphics[width=\textwidth]{figures/appendix/appendix_DRAEM/SB/144_mask.png}
        \end{minipage}
        \begin{minipage}{0.45\textwidth}
            \centering
            \includegraphics[width=\textwidth]{figures/appendix/appendix_DRAEM/SB/337.png}
            \includegraphics[width=\textwidth]{figures/appendix/appendix_DRAEM/SB/337_m.png}
            \includegraphics[width=\textwidth]{figures/appendix/appendix_DRAEM/SB/337_mask.png}
        \end{minipage}
    \end{subfigure}

    \begin{subfigure}[b]{\textwidth}
        \centering
        \begin{minipage}{0.45\textwidth}
            \centering
            \includegraphics[width=\textwidth]{figures/appendix/appendix_DRAEM/SC/175.png}
            \includegraphics[width=\textwidth]{figures/appendix/appendix_DRAEM/SC/175m.png}
            \includegraphics[width=\textwidth]{figures/appendix/appendix_DRAEM/SC/175_mask.png}
        \end{minipage}
        \begin{minipage}{0.45\textwidth}
            \centering
            \includegraphics[width=\textwidth]{figures/appendix/appendix_DRAEM/SC/237.png}
            \includegraphics[width=\textwidth]{figures/appendix/appendix_DRAEM/SC/237m.png}
            \includegraphics[width=\textwidth]{figures/appendix/appendix_DRAEM/SC/237m.png}
        \end{minipage}
    \end{subfigure}


    \caption{Example segmentations of the DRAEM \cite{Zavrtanik_2021DRAEM} approach on the MVTecAD LOCO \cite{LOCODentsAndScratchesBergmann2022} dataset.}
    \label{fig:appendixDRAEM}
\end{figure}

            % insert appendix here if needed

\end{document}

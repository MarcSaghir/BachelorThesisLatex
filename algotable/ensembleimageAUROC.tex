\begin{table}[htbp]
    \tiny
    \centering
    \begin{tabularx}{\textwidth}{|X|X|X|X|X|X|X|X|}%{|c|p{5cm}|p{5cm}|p{5cm}|}
        \hline
        \textbf{Method} & \textbf{Breakfast Box} & \textbf{Flat Connector} & \textbf{Juice Bottle} & \textbf{Pushpins} & \textbf{Scew Bag} & \textbf{Splicing Connectors} & \textbf{Average} \\
        \hline
        WR50L23 + R50L23 + R18L23  & 0.698 & 0.931 & 0.722 & 0.617 & 0.621 & 0.632 & 0.844 \\
        \hline
        WR50L12 + WR50L23 & 0.745 & 0.989 & 0.760 & 0.618 & 0.666 & 0.643 & 0.884 \\
        \hline
    \end{tabularx}
    \caption{Ensemble results on MVTecAD LOCO \cite{LOCODentsAndScratchesBergmann2022} by this works introduced ensemble approaches. The leftmost columns represents the ensemble 
             constellation. The abbreviations denote the following: 
             WR50 = wideresnet50, R50 = resnet50, R18 = resnet18. Lxy means an aggreation of layers x and y. Lastly the + sign shows that the listed abbreviations belong to one ensemble.}
    \label{tab:ensembleimageAUROC}
\end{table}





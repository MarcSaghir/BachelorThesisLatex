\begin{abstract}

The field of image anomaly detection (IAD) has been very popular in recent years. This development can largely be attributed to its application as quality control in manufacturing 
settings, such as industrial ones. It has had many breakthroughs already. State-of-the-art classifiers achieve classification and localization performance of up to $99.6 \%$ 
on various datasets. However, those results were produced in very clinical settting and on structural anomalies, such as a surface with a scratch. A different crucial 
type of anomalies are logical anomalies. They constitute violations of rules, such as wrong object orientations or missing parts. Current anomaly detection classifiers 
show a significantly worse performance on this second anomaly type, even though they are just as common and potentially vital for a more varied quality control process 
in real-world settings.\newline
To enhance robustness, we propose a feature-level ensemble pipeline that integrates diverse feature representations and existing anomaly detection architectures. 
By utilizing feature representations from multiple extractors 
and projecting them into a unified space, we aim to provide increased information, thereby improving precision and robustness in anomaly localization. Additionally, 
we conduct a survey of representative IAD approaches on a novel dataset that includes logical anomalies, offering an extensive overview of current methods and their 
potential for logical anomaly localization. We also introduce a new dataset category, extending the primary dataset used in this study.\newline

Our experiments confirm that the performance of classifiers significantly decreases when both structural and logical anomalies are considered, validating our initial 
assumptions. We discuss specific characteristics and cross-classifier comparisons in detail. However, our ensemble approach did not enhance performance or robustness. 
The results were significantly inferior to those of individual ensemble members. This outcome highlights the need for further research into feature-level ensembling methods. 
Despite not presenting extraordinary difficulty, the new dataset category provided valuable insights into classifier performance and serves as a valid extension of the 
logical dataset.

\end{abstract}
%
%
%\selectlanguage{ngerman} % select german language
%\begin{abstract}[2]
%	Hier können Sie Ihre deutsche Zusammenfassung schreiben. %
%\end{abstract}
%\selectlanguage{english} % reset to english language

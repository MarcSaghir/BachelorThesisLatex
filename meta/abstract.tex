\begin{abstract}

Image anomaly detection (IAD) has been very popular in recent years. This development can largely be attributed to its application as quality control in manufacturing 
settings, such as industrial ones. It has had many breakthroughs already. State-of-the-art classifiers achieve a classification and localization performance of up to $99.6 \%$ 
on various datasets. In IAD, there is a distinction between two types of anomalies: structural and logical. Structural anomalies denote material damage to the class object, while logical ones 
comprise anomalies that violate a set of rules and are not necessarily constrained to one region. \newline 
Current research in anomaly detection has predominantly focused on evaluating classifiers solely on structural anomalies. Moreover, popular dataset images in this area are often 
in a clinical setting. This approach is problematic as it neglects the detection of logical anomalies that are often critical in real-world applications. This results in 
IAD classifiers, with initially robust and high performance, to significantly drop in detection and localization quality. \newline
To enhance robustness, a feature-level ensemble pipeline is proposed. It integrates diverse feature representations and existing anomaly detection architectures. 
By utilizing feature representations from multiple extractors 
and projecting them into a unified space, the ensemble aims to provide increased information, thereby improving precision and robustness in anomaly localization. Additionally, 
this work conducts a survey of representative IAD approaches on a novel dataset that includes logical anomalies, offering an extensive overview of current methods and their 
potential for logical anomaly localization. Furthermore, a novel dataset category is introduced, extending the primary dataset used in this study. It features both logical and structural anomalies.\newline
Our experiments confirm that classifiers' performance significantly decreases when structural and logical anomalies are considered, validating our initial 
assumptions. Specific characteristics and cross-classifier comparisons are discussed in greater detail. However, our ensemble approach did not enhance performance or robustness. 
The results were significantly inferior to those of individual ensemble members. This outcome highlights the need for further research into feature-level ensembling methods. 
Despite not presenting extraordinary difficulty, the new dataset category provided valuable insights into classifier performance and served as a valid extension of the 
logical dataset.

\end{abstract}
%
%
%\selectlanguage{ngerman} % select german language
%\begin{abstract}[2]
%	Hier können Sie Ihre deutsche Zusammenfassung schreiben. %
%\end{abstract}
%\selectlanguage{english} % reset to english language

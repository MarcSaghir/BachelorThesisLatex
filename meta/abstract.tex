\begin{abstract}

Image anomaly detection (IAD) has been very popular in recent years. This circumstance can largely be attributed to its application as quality control in manufacturing 
settings, such as industrial ones. It has had many breakthroughs already. State-of-the-art classifiers achieve a classification and localization performance of up to $99.6$ percent 
on various datasets. In IAD, there is a distinction between two types of anomalies: structural and logical. Structural anomalies denote material damage to the class object, like 
a scratch, while logical ones 
comprise anomalies that violate a set of rules and are not necessarily constrained to one region. An example of a logical anomaly would be a missing object or one pointing in a 
direction not permitted by the rule set of the class. \newline 
Current research in anomaly detection has predominantly focused on evaluating classifiers solely on structural anomalies. Moreover, popular dataset images in this area are often 
in a clinical setting. This approach is problematic as it neglects the detection of logical anomalies that are often critical in real-world applications. This results in 
IAD classifiers, with initially robust and high performance, to significantly drop in detection and localization quality. This drop in performance may range from six to 
twenty-nine percent, which is significant considering the previously mentioned high performance.\newline
To enhance robustness, a feature-level ensemble pipeline is proposed. It integrates diverse feature representations and existing anomaly detection architectures. 
By utilizing feature representations from multiple extractors 
and projecting them into a unified space, the ensemble aims to utilize increased information, thereby improving precision and robustness in anomaly localization. Additionally, 
this work conducts a survey of representative IAD approaches on a novel dataset that includes logical anomalies, offering an extensive overview of current methods and their 
potential for logical anomaly localization. Furthermore, a novel dataset category is introduced, extending the primary dataset used in this study. It features both logical and 
structural anomalies.\newline
The experiments confirm that classifiers' performance significantly decreases when structural and logical anomalies are considered, validating initial 
assumptions. Specific characteristics and cross-classifier comparisons are discussed in greater detail. However, the ensemble approach did not enhance performance or robustness. 
The results were significantly inferior to those of individual ensemble members. Further investigations have shown that the root cause for this lies within the method of 
feature combination.
This outcome highlights the need for further research into feature-level ensembling methods, especially in this context. 
Despite not presenting extraordinary difficulty, the new dataset category provided insights into classifier performance and served as a valid extension of the 
logical dataset. Future studies regarding this topic may utilize this dataset class as the base for a new set of logical anomalies to further expand the variety of logical 
datasets for IAD.

\end{abstract}
%
%
\selectlanguage{ngerman} % select german language
\begin{abstract}%[2]
%	Hier können Sie Ihre deutsche Zusammenfassung schreiben. %

Die Erkennung von Bildanomalien ist in den letzten Jahren sehr beliebt geworden. Dieser Umstand ist vor allem auf die otentielle Anwendung als Qualitätskontrolle 
in industriellen Fabriken zurückzuführen. Es hat bereits viele Durchbrüche in den letzter Zeit gegeben. Aktuelle Modelle erreichen eine Klassifizierungs- und 
Lokalisierungsleistung von bis zu $99,6$ Prozent auf verschiedenen Datensätzen. In dem Gebiet wird zwischen zwei Arten von Anomalien unterschieden: strukturelle und 
logische. Strukturelle Anomalien bezeichnen materielle Schäden am Objekten, wie ein Kratzer, während logische Anomalien 
solche umfassen, die vordefinierte Regeln verletzen oder nicht potentiell mehrere Regionen überspannen. Ein Beispiel für eine logische Anomalie wäre ein fehlendes 
Objekt oder eines das in eine Richtung zeigt, die nach den Klassenregeln nicht zulässig ist. \newline
Die derzeitige Forschung auf dem Gebiet der Anomalieerkennung konzentriert sich hauptsächlich auf die Bewertung von Modellen, die ausschließlich auf strukturelle 
Anomalien basieren. Darüber hinaus sind die Inhalte der gängigen Datensätze in diesem Bereich oft sehr künstlich hergestellt worden. Dies ist problematisch, da es die 
Erkennung von logischen Anomalien vernachlässigt, die in realen Anwendungen oft einen relevanten Mehrwert bringen. Dies führt dazu, dass
Klassifikatoren, die anfangs robuste und hohe Leistungen erbringen, erheblich an Erkennungs- und Lokalisierungsqualität verlieren. Dieser Leistungsabfall kann zwischen 
sechs und neunundzwanzig Prozent betragen, was in Anbetracht der zuvor erwähnten hohen Leistung beträchtlich ist.\newline
Um die Robustheit dieser Modelle zu verbessern, wird hier eine Ensemble-Pipeline auf Merkmalsebene vorgestellt. Sie integriert verschiedene Merkmalsrepräsentationen und 
bestehende Architekturen zur Erkennung von Anomalien. 
Durch die Verwendung von Merkmalsrepräsentationen aus mehreren Extraktoren 
und deren Projektion in eine vergleichbare Repräsentation, zielt das Ensemble darauf ab, mehr Informationen einzubeziehen und dadurch die Präzision und Robustheit bei der 
Lokalisierung von Anomalien zu verbessern. Zusätzlich verschafft diese Arbeit einen Überblick über repräsentative Anomalieerkennungsansätze auf einem neueren Datensatz, welcher 
logische Anomalien enthält, und bietet eine Analyse bezüglich deren 
Potenzial für die Lokalisierung logischer Anomalien. Darüber hinaus wird eine neue Datensatzkategorie eingeführt, die den in dieser Studie verwendeten Datensatz erweitert. 
Sie umfasst sowohl logische als auch strukturelle Anomalien. \newline

Experimente bestätigen, dass die Leistung der Modelle deutlich abnimmt, wenn sowohl strukturelle als auch logische Anomalien berücksichtigt werden, was die 
ursprünglichen Annahmen bestätigt. Spezifische Merkmale und Vergleiche zwischen den Klassifikatoren werden ebenfalls ausführlicher erörtert. Der Ensemble-Ansatz hat 
jedoch weder die Leistung noch die Robustheit verbessert. 
Die Segmentierungen waren deutlich schlechter als die der einzelnen Ensemblemitglieder. Weitere Untersuchungen haben gezeigt, dass die Ursache hierfür in der Methode der 
Merkmalsvereinigung liegt.
Dieses Ergebnis unterstreicht die Notwendigkeit weiterer Forschung zu Methoden der Merkmalsvereinigung, insbesondere in diesem Zusammenhang. 
Obwohl die neue Datensatzkategorie keine außergewöhnlichen Schwierigkeiten für dir Modelle aufwies, lieferte sie sinnvollen Einblicke in die Klassifikatorleistung und diente 
als gelungene Erweiterung des 
logischen Datensatzes. Künftige Studien zu diesem Thema könnten diese Klasse als Grundlage für einen neuen Satz logischer Anomalien verwenden, um die Vielfalt der logischen 
Datensätze für IAD zu erweitern.



\end{abstract}
\selectlanguage{english} % reset to english language

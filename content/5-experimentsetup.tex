\chapter{Experiments}
\label{chap:experiments}

In this chapter the experiments and experimental results of this work are displayed. First we establish the general experimental setup. Afterwards the results of the conducted MVTecAD LOCO \cite{LOCODentsAndScratchesBergmann2022}
survey are presented in section \ref{sec:locoxperiments}. As a baseline for performance evaluation we uitilize the performance on the more conventional MVTecAD dataset \cite{MVTEC_Bergmann_2021}, 
as well as a classifier comparison. In section \ref{sec:faltconnectorxperiments} we review the performance of the classifiers on our novel dataset category. Lastly section \ref{sec:ensembleresults} 
deals with the findings regarding the ensemble network approach, also conducted on the MVTecAD LOCO dataset.


\section{Experimental Setup}
\label{sec:experimentsetup}

All models trainings and result reproductions have been conducted on the IAD cluster student partition. The GPU in use for all nodes used by that partition (ist die formulierung gut?) is and 
RTX 2080Ti with 11GB of memory, and the CPU is an AMD Ryzen 9 16-Core processor. The cluster overview \cite{clusterdocs} serves to provide further detail for additional questions. As for software, 
pytorch 2.1.2 was utilized to implement the ensemble model. The specifications of other libraries, as well as the specifications for the MVTecAD LOCO experiment are documented in the 
environment files in the project code.



\section{MVTecAD LOCO Experiments}
\label{sec:locoxperiments}
In this section we review the performance of prior introduced anomaly detection methods. All experiments were performed with the same 
conditions explained in section (referenz von methods section über loco) 
and on the MVTecAD LOCO dataset \cite{LOCODentsAndScratchesBergmann2022}. 
The results of inference on the test set can be seen in table x (tabelle mit ergebnissen). As it can be seen, all models scored a significantly 
lower result on the MVTecAD LOCO dataset than on the normal MVTecAD one(exemplary scores seen in table xy(table mit normalen mvtec scores)). 
A lower performance is generally to be expected, since firstly logial anomalies are regarded as a more difficult problem than structual 
ones and secondly the average SOTA performances as seen in table x(tabelle mit ergebnissen) is already closing in on an AUROC of of 1. 
(den satz rechts von hier müsste man maybe rausmachen oder umschreiben)Therefore there is not much room for further improvement in similar settings, and a worse performance still aknowledgeable as very good. 
Yet there is an drop in cross-model average AUROC of approcimately (durchschnitts drop ausrechnen), which is a remarkable(synonym) difference. 
Most other metrics, namely (metrics names), also declined with an respective average of (respective averages). As explained in section 
(referenz zu metrics section von background), the sPRO (or rather AU-sPRO) was a score introduced in \cite{LOCODentsAndScratchesBergmann2022} to gain an 
advanced insight on the quality of segmentations. This means that all approaches who either were published before or did not include this 
paper in their research likely did not include this metric, which holds true for the approahces used for this experiment. Therefore no comparison 
in sPRO/AU-sPRO can be shown(vllt einfach spro auch für allte ansätze implementieren?? dann kann ich den satz ändern). Comparing the sPRO 
scores of the SOTA methods in this experiment with the ones from compared to GCAD \cite{LOCODentsAndScratchesBergmann2022} shows asignificantly 
(abchecken ob wirklich) worse performance.
Among the different models, the highest scoring one was PatchCore \cite{patchCore2022}. It scored an average (metrics einfügen) feature embedding based approaches like  
achieved the highest scoring

Interpretation of results hier, weiß nicht in welche section das eigentlich muss:


\section{Flat Connector Experiments}
\label{sec:faltconnectorxperiments}


\section{Ensemble Network}
\label{sec:ensembleresults}





\chapter{Conclusion and Outlook}
\label{chap:conclusion}


After now having experimented on concepts made known in chapter \ref{chap:method}, reviewed the results and interpreted them, it is 
fitting to draw a conclusion from this work. This chapters focus is firstly on what to take away and secondly giving thought to 
possible ablation studies or new concepts that could be researched on the basis of this paper.


\section{Conclusion}
\label{sec:realconclusion}

\textbf{IAD Approaches on MVTecAD LOCO.}\newline
The survey on logical anomaly detection and localization using recent state-of-the-art IAD approaches went as expected. There was a performance drop, that could at least 
partially be traced back to the addition of logical anomalies to the experiments. Moreover the classifiers all exhibited reasonable sPRO metrics that were predictable for the 
other metrics. The results and discussion sections confirm that these approaches have clear need for improvement when shifting the focus from the classic MVTecAD \cite{MVTEC_Bergmann_2021} 
to a more challenging dataset containing logical anomalies. This underlines one of the problems mentioned in the introduction, namely poor robustness for more complex anomalies. \newline
Judging from the segmentation results, these approaches have potential to be adjusted for better logical anomaly localization performance. This could either be through 
further research for improved approaches, or combination of approaches to utilize different strenghts. Examples of this are rpesented in the outlook.

\textbf{Flat Connector Class.}\newline
The novel dataset category is usable and allows for further interesting insights into various IAD methods. It's difficulty compared to other logical anomaly classes may be 
below average. Yet this does not disqualify it from being a viable dataset extension. Having classes from multiple difficulties in one set helps evaluate classifiers on 
multiple levels, as well as giving opportunity for performance evaluation to strong and weaker approaches of IAD. Lastly it extends the previously existing logical anomaly 
concepts, providing a more diverse experiment setting. \newline
The class may benefit from being extended regarding kinds of anomalies, as well as more training and testing images. Ideas on how to potentially grow this class are given 
in the outlook section of this chapter.

\textbf{Ensemble.}\newline
The ensemble approach presented in the previous chapters did not yield an improvement in robustness, or any improvement in performance at all. The segmentations were either 
worse or unusable. The feature ensembling approach by Heller et al. \cite{EnsembleHeller2023} did not work in this setting. Results from applying it were not showing any learning 
process (Fig. \ref{fig:pca_res}) and possible reasons were discussed in section \ref{subsec:ITBfaildiscussion}. This calls for further experimentation with this approach as mentioned 
in the outlook. Stacking the channels to create merged feature maps did produce results that displayed potential. Still the base ensemble methods had a better localization 
performance. On this basis, it is to be said that merely feature map stacking is not a promising procedure for this problem statement, and should only be used in combination 
with other methdos to reduce and combine the number of channels.

\textbf{Current State of the Pipeline.}\newline
After the conducted experiments, there now remains an code structure to be used for potential further experiments. 
All approaches discussed in this work are contained within the structure and their functionality is adapted to experiment on the MVTecAD LOCO \cite{LOCODentsAndScratchesBergmann2022} dataset. Most 
importantly, there also exists an ensembling pipeline that allows ensemble members, as well as ensembling methods to be switched modularly. The ensembling members and methods 
serve es a two part black box, both of which are likely root causes of failing ensemble performances. 
Through exchanging components in these regions, ablation studies can easily be performed on this topic, and the goal of achieving greater anomaly localization robustness 
could be realized with a different approach.


\section{Outlook}
\label{sec:finaloutlook}

After concluding the findings of this work it is imperative to analyze aspects that could have been done differently, or discuss new possible research that could emerge from 
the basis of this work as well as related work.\newline

\subsection{Current IAD Methods}
The conclusions drawn from the survey were reported in the previous section. The decreasing performance trend for logical anomaly detection in overall performance suggests 
current IAD approaches to be unsuitable to reach the same effectiveness as on structural ones. This calls for a shift in fundamental perspective of visual anomaly 
detection methods, tailoring them towards logical anomaly detection as in \cite{LOCODentsAndScratchesBergmann2022}, or try to improve them by combining them to raise their 
performance. Solutions to approach this problem could consist of combining 
those state of the art methods into potentially more effective ones, by using the upsides of multiple IAD methods to produce new approaches. An example for this would be a 
merger of the PatchCore \cite{patchCore2022} and SimpleNet \cite{liu2023simplenet} methods. Using a feature adapter, trained in the latter approach, to project feature 
representations before adding them to a memory bank could constitute for an interesting step in this direction. 

\subsection{Novel Dataset Classes}
Regarding the novel dataset category introduced in chapter \ref{chap:datasets}, it was apparent that the performance of all approaches was significantly higher than for other classes. 
As this may be an indicator for the class to be to easy, it is key to look forward on how to either modify the class, or introduce new ones that have similar representation in a 
manufacturing context. \newline
With respect to current implemented logical anomalies of the flat connector class, it may be resource extensive but worth to professionally produce anomalous flat connectors. This 
could cause more real looking anomalies and even help create new ones. Exemplary one could shift the position of a hole as a new logical anomaly category. Alternatively the classes 
of missing holes could be produced free of artifacts, potentially leading to new results and more precise conclusions. Another way to extend the anomalies introduced for this 
object would be to work with arbitrary logical rules for collections of flat connectors. This could include presenting a grid of flat connectors and detecting missing ones or falsely 
oriented ones, similar to the pushpin class of the MVTecAD LOCO \cite{LOCODentsAndScratchesBergmann2022} dataset.\newline 
Since the effort to extend the logical anomaly dataset is ongoing, there are numerous possibilities on how to do so. Instances may include a class consisting to a metal angle. 
Here it would be a noteworthy opportunity to construct this class as a multiperspective dataset, as the angle may have different orientations and ambiguous positions. Depending 
on the orientation of the angle on the surface, one camera might not be able to capture every aspect of the three dimensional object. An approach could be to learn several models 
on different perspectives and ensemble classifier outputs or even feature representations afterwards.

\subsection{Ensemble}
\label{subsec:ensembleconc}

\textbf{Backbones.} An outlook on backbones used for this works ensemble approach is to conduct more experiments with either other backbones or a higher variety of hierarchies. 
This may be conducted in either the same way as done here or with potentially different architectures.
\newline\newline
\textbf{Ensembling Method.} The discussion made it apparent that the independent transformation block \cite{EnsembleHeller2023} or at least PCA is not applicable in this context. 
Still, there may be solutions to more efficiently combine different feature representations. In the case that the PCA did not work as intended on the features due to non linear 
relationships of the variables, there are methods to overcome that problem. They include methods like t-SNE\cite{tSNE} or kernel PCA \cite{Hoffmann_2007kernelPCA}, both potentially 
viable methods for this problem. Furthermore there exist possibilities outside of standard dimensionality reduction that may be experimented on. Possible approaches may include 
utilizing networks similar to the feature adapter approach from Simplenet (section \ref{subsec:simplenet}) to ensemble features channel wise.
\newline\newline
\textbf{Feature Representations.} Lastly, as mentioned in the conclusion, the current ensemble network is currently a pipeline that consists of two black boxes, one for the ensemble 
members feature representations and one for the chosen ensemble methods. The latter is covered in the last subsection. For the box of feature representations there are ablation 
experiments to be done using different IAD approaches. One could introduce new appropriate feature representations from different IAD methods to input to the ensemble pipeline for 
further studies.


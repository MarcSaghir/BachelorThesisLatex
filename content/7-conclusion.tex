\chapter{Conclusion and Future work}
\label{chap:conclusion}

\section{Ensemble Network}
\label{sec:ensembleconclusion}

- einleitungssatz dass hier die ensemble analyse für den flachverbinder kommt

\subsection{Independent Transformation Block}
\label{subsec:ITBfailconc}

- sagen dass analyse ergeben hat dass pca hier gefailt hat. testweise wurde auch pca auf nur die simplenet features angewandt, und zeigte ähnliche ergebnisse.
- das heißt fehler liegt wahrscheinlich bei pca
-> gründe nennen, zB: pca ist nicht immer anwendbar, nur bei linearen dependencies
-> belegen mit quelle wo grenze von PCA ist
- sagen was potentiell getan werden kann, zb tSNE und Kernel PCA als verfahren nennen um so probleme zu überbrücken plus quellen


\subsection{Stacking Ensemble}
\label{subsec:stackingconc}

\section{SOTA performance}
\label{sec:sotaperformanceconc}

\section{Flat connector}
\label{sec:flatconnectorconc}

- performance von den ansätzen auf flat connector bewerten
- vergleiche zum restlichen loco dataset ziehen bzgl performance
- wenn ich das gemacht habe dann auch vergleiche zwischen strukturellen und logischen anomalien performance machen

- schlussendlich sagen ob die klasse sich gelohnt hat, oder zu einfach war und welche aspekte sehr herausfordernd waren(zb der loch größen unterschied)

\section{Outlook}
\label{sec:outlook}


- schreiben was es noch für coole dataset categories gab -> multiperspective angle und blech konstruktion mit schrauben

subsection ensemble network
- schreiben dass man an der feature ensembling methode arbeiten kann, zb halt mit tsne oder kernel PCA da neue sachen probieren kann, sonst auch anderen methoden außer stacking(beispiel)
- man kann die layer der verschiedenen backbones optimieren plus die kombi
- schreiben die einbindung von andern approaches ins ensemble wie das füttern von patchcore mit diesen features
- generelles hinzufügen von anderen feature approaches weil ensemble ja blackbox mäßig ist.




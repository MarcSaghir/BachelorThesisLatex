\chapter{Introduction}
\label{chap:introduction}

Image Anomaly detection (IAD) as a form of quality control is a widely prevalent practice in modern manufacturing processes. In the last 74 years, the steel production worldwide, 
has increased by more than ninefold \cite{worldsteel} due to innovation and the growing usage of metal in modern applications. 
With these developments came a high need for quality 
assurance alongside raised standards and requirements. These strict conditions serve, among other things, to avoid product failure that could cause fatal consequences. Quality control in the form of anomaly detection often 
starts with the individual parts manufactured for a single purpose, which demands considerable effort and lots of resources due to factors like 
the above-mentioned increasing production rate. 
In earlier days, this meant procedures like manual stochastic quality checks of produced parts, a practice that cannot give 
complete certainty. It is very labor-intensive and, therefore, comes at a high price point for the manufacturers. Later, with the rise of computers and sophisticated computer vision methods, 
this quality control process became increasingly automated, with methods like IAD ensuring a more efficient quality control procedure. With time, these developments, 
alongside a constant striving of industries towards even higher reliability and later the recent developments in artificial intelligence, brought forth IAD as the 
well-researched research field it is today.\newline 
In our context, IAD is a subcategory of general anomaly detection 
that aims to distinguish images of a category that conform to some chosen norm from anomalous images of the same category that do not. 
An example would be creating a classifier that is given the image of a screw and can detect whether it conforms to our expectations, 
which in a manufacturing setting likely means meeting the company's quality standards(Fig. \ref{fig:introscrew}).

\begin{figure}[htbp]
    \captionsetup[subfigure]{justification=centering}
    \centering
    \begin{subfigure}[b]{0.28\textwidth}
        \centering
        \includegraphics[width=\textwidth]{figures/introductionanomalies/033.png}
        %\caption*{original Image}

    \end{subfigure}
    \hspace{0.05\textwidth} % Add space between subfigures
    \begin{subfigure}[b]{0.28\textwidth}
        \centering
        \includegraphics[width=\textwidth]{figures/introductionanomalies/manifrontcopy.png}
        %\caption*{Mask}

    \end{subfigure}
    \hspace{0.05\textwidth} % Add space between subfigures
    \begin{subfigure}[b]{0.28\textwidth}
        \centering
        \includegraphics[width=\textwidth]{figures/introductionanomalies/realfrontnow.png}
        %\caption*{Discriminator Predictions}

    \end{subfigure}
    \caption{Instance images of a screw from \cite{MVTEC_Bergmann_2021} to showcase anomalous objects in manufacturing settings. The images include a regular screw, a screw with a chipped head and a screw with a bent tip.}
    \label{fig:introscrew}
\end{figure}

With IAD being a very recent and popular field, various deep-learning approaches have established themselves over 
the last few years. The best-performing ones have generally been unsupervised learning approaches. This circumstance stems from the fact that 
in any manufacturing setting, far fewer anomalous parts exist than regular ones, which creates a significant data imbalance. 
Moreover, obtaining a large number of data points and great variance can be difficult since it is much work to 
coordinate with adequate manufacturing facilities and implement the necessary infrastructure to take pictures. This problem is supported 
by the fact that few well-established, wide-ranging datasets are being used for modern IAD research. There are still some credible and widely used 
datasets, amongst them the MVTecAD \cite{MVTEC_Bergmann_2021} dataset acting as a gold standard. The dataset will be discussed in greater detail 
in the background chapter. Regarding the kind of unsupervised anomaly detection methods, a great variety of approaches follow a 
distinct strategy of differentiating between the classes. Still, most of them can be categorized into two classes: representation- 
or reconstruction-based methods. While representation approaches aim at creating a feature representation to compare the features of new input images then, reconstruction-based ones try to learn how to recreate the part shown in the image as an anomaly-free object and then compare the constructed product to the original input. Both workflows are visualized in figure \ref{fig:vizofrecrepbased}, which showcases 
the individual steps of the respective methods as described. It is to be said that both approaches offer high-quality predictions, yet 
feature representation methods have shown more frequently in latest research to achieve state-of-the-art results \cite{liu2024deep}. % \cite{Xie_2024benchmarking}.
\newline
The current state of IAD generally consists of very high-performing classifiers. Here, it is essential to differentiate between the  
applications of those classifiers. There is anomaly detection in the form of image classification, which has already been mentioned. 
Furthermore, there is anomaly localization, referring to the image segmentation process to point out the specific regions where 
the detected anomaly occurs. Lastly, besides the applications, one can also categorize anomalies. The most researched anomaly types 
are so-called structural anomalies, which can be described as superficial damages of the part's material or shape, i.e. a strongly 
bent screw or one that is broken in the middle. However, recently, there has been a new dataset from the creators of MVTecAD that covers logical 
anomalies, namely the MVTecAD LOCO \cite{LOCODentsAndScratchesBergmann2022} dataset. Logical anomalies denote ones that violate an abstract 
set of rules. More concretely, this can mean instances like a metal part with an irregular number of holes or a label missing. Whereas 
state-of-the-art approaches produce performance metrics of up to $99.6 \% $ on classifying structural anomalies, they 
strongly diverge in anomaly localization performance. Moreover, the performance plummets when approaching the classification and localization of logical 
anomalies. Additionally, models often show inconsistencies between individual subtypes of structural and logical anomalies, especially during 
localization. Another important consideration is that most anomaly detection approaches have distinct weaknesses, which worsen most 
of their performances in certain situations. These inconsistencies and performance gaps demonstrate that IAD, as such, is not yet solved and still requires improved 
robustness and generalizability. This need also holds true due to logical anomalies making up an important new domain of automated 
quality control, as more complex parts could be tested for requirements. Moreover, showcasing performance inconsistencies between 
structural and logical anomalies indicates the latter being of a different problem domain. Achieving better translation between those 
settings could serve as a basis for tackling other problems in this field that may arise in future settings.


\section{Contributions}
\label{sec:contributions}
This research provides multiple contributions to the field of image anomaly detection to provide a structured overview of current IAD research and 
further, push the progress of robust anomaly localization. 

\begin{enumerate}
\item To address the problems mentioned at the end of the last section, we attempt 
to build a feature-level ensemble network, combining various feature representations to improve 
general performance alongside robustness in logical anomaly localization and detection. This ensemble network is then tested on 
the MVTecAD LOCO dataset \cite{LOCODentsAndScratchesBergmann2022} to observe its performance regarding both anomaly types.
\item Furthermore, an extensive survey is conducted on the performance of a diverse selection of IAD methods on the MVTecAD LOCO dataset. 
This study highlights the current state of anomaly detection concerning logical problems and investigates the application potential of 
those approaches a transfer learning setting.
\item Second to last, we introduce a new category to the MVTecAD LOCO dataset to further increase the dataset's diversity and strengthen this thesis's focus on metal-manufactured parts. Many datasets either use synthetic data or images in a very clinical setting. Therefore, this 
attempt for variance is also a step towards IAD for more realistic datasets.
\item Finally, the mentioned network and experiments are also streamlined into an easy-to-use pipeline 
for future experiments in that area.
\end{enumerate}


The contributions mentioned firstly benefit faster research entry and an accelerated experimentation process, with an intuitive setup, 
as well as potential industrial applications. 
Furthermore, they give more insight into the capabilities of existing methods in an industrial setting and thus also provide a more 
diverse and practical setting than the prior categories in the MVTecAD \cite{LOCODentsAndScratchesBergmann2022} dataset. The same methods are also tested on their limitations 
regarding logical anomalies, which were made to be a relevant aspect of anomaly detection in current manufacturing quality control 
settings. Lastly, the use of a robust ensemble approach for potentially heterogeneous classifiers opens up possibilities for expanding 
the range of application of state-of-the-art IAD methods to other domains with robust performance. It may also produce more usable results in real-world IAD settings. The presented network can also be used as a foundation for future experiments in 
various directions. Possible outlooks based on this kind of multi-feature representation ensembles are discussed in section \ref{sec:finaloutlook}.
\newline
\newline

In this work, we will first discuss relevant background knowledge to grasp the latest significant IAD research and fundamental principles relevant to the ensemble 
approach presented here. Section \ref{sec:IADcategs} and section \ref{sec:IADmethods} give a broad overview of state-of-the-art IAD approaches and intuitive knowledge on how to 
view them. Afterwards, sections \ref{sec:metrics} and \ref{sec:datasets} offer insight into the testing and evaluation environments of this context. Lastly, section \ref{sec:ensembles} and 
section \ref{sec:modelcalibration} focus on background knowledge of common approaches regarding the ensemble model discussed before.\newline
The third chapter presents the novel dataset category introduced in this paper.
In chapter \ref{chap:method}, we present the concrete implementation of this work's contributions, starting with section \ref{sec:lcocsurveymethods} dealing with the survey on logical anomalies. 
Section \ref{sec:ourensemblenetwork} concerns the realization of this ensemble approach.
\newline 
Chapters \ref{chap:results} and \ref{chap:conclusion} afterwards deal with an in-depth analysis of our findings as well as an outlook on how to interpret these results and what future research 
in this topic may be.







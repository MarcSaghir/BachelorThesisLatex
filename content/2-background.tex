\chapter{Background}
\label{chap:background}
This is an algorithm 





\section{Classes of Anomaly detection}
- there are different kinds of approaches to IAD
- look at tree picture
 
- First important distinction is between supervised and unsupervised
-> we focus on unsupervised
-> list problems with supervised approaches and thus advantages of unsupervised ones

- briefly touch on other IAD settings like few shot, along with references

- among unsupervised approaches, there are two more fundamental distinctions
-> reconstruction based vs representation/feature embedding based
-> explain difference with lots of references

- for reconstruction based touch on 2-3 base categories like GANs etc and link fundamental papers for GANs etc
- for representation based important to explain memory bank, teacher student, and distribution map
- explain normalizing flow somehow somewhere in there

- maybe say which algos we chose and what we covered with that


\section{Our own Dataset}
- repeat motivation why we added additional data in mvtec style
- say that we went with loco mvtec flair(maybe give reasons)
- say that we came up with a set of structural and logical anomalies for each category
- list categories(flat connector, angle and special construct)

- 3 sub sections for the three categories

- flat connector
- link the exact one we used(or examples of some)
- give structural anomalies
- give logical anomalies
- for both briefly touch on how we produced them
- show image examples for each

- repeat same for other categories

- also when describing angle:
- touch on how there is a special case with multi perspective detection



\section{metrics}

- show metrics from survey papers
- explain which metrics we used and where the other ones are used
- explain also why we used the ones we used, and what disadvantages of other ones where

- touch on paul bergmann paper for sPRO score, say how it is better than pixel auroc and normal pro score, also explain saturation thresholds

- some math formula for calculating the important metrics






















{\SetAlgoNoLine%
\begin{algorithm}[H]
\SetKwInOut{Input}{Input}
\SetKwInOut{Output}{Output}
\SetKwFor{For}{for}{do}{end for}
\SetCommentSty{small}
\DontPrintSemicolon
% KwIn
\textbf{Input: }$\theta_1, \theta_2, \phi$ \tcp*[r]{Initial parameters}
\Indp % einruekung
$\bar{\theta_1} \leftarrow \theta_1, \bar{\theta_2} \leftarrow \theta_2$\tcp*{Initialize target network weights}
$\mathcal{D}$ \leftarrow \O \tcp*[r]{Initialize an empty replay pool}
\ForEach{iteration}{
    \ForEach{environment step}{
        $a_t \sim \pi_{\phi}(a_t|s_t)$ \tcp*[r]{Sample action from the policy}
        $s_{t+1} \sim \p((s_{t+1} |s_t, a_t)$ \tcp*[r]{Sample transition from the environment}
        $\mathcal{D} \leftarrow \mathcal{D} \{(s_t, a_t, r(s_t, a_t), s_{t+1}  )\}$ \tcp*[r]{Store the transition in the replay pool}
        }
    \ForEach{gradient step}{
        $\theta_i \leftarrow \theta_i - \lambda_{\mathcal{Q}} \hat{\nabla}_{\theta_i} J_{\mathcal{Q}}(\theta_i)$ for $i \in \{1,2\} $\tcp*[r]{Update the Q-function parameters}
        $\phi \leftarrow \phi - \lambda_{\pi} \hat{\nabla}_{\phi} J_{\pi}(\phi)$\tcp*[r]{Update policy weights}
        $\alpha \leftarrow \alpha - \lambda \hat{\nabla}_{\alpha} J_{\alpha}(\phi)$\tcp*[r]{Adjust temperature}
        $\bar{\theta_i} \leftarrow \tau \theta_i + (1-\tau) \bar{\theta}_i$ for $i \in \{1,2\}$ \tcp*[r]{Update target network weights}
        }
} 
\Indm  % undefined control seq error, but it compiles
\textbf{Output: }$\theta_1, \theta_2, \phi$ \tcp*{Optimized parameters}
\caption{Soft Actor Critic}
\label{algorithm:sac}
\end{algorithm}%
}
\chapter{Discussion}
\label{chap:results}

- describe function of this chapter\newline

at reader: I have some text/images for this but it is very scattered. Therefore i firstly just included the bullet points so you can comment on the contents.\newline


\section{Current Approaches Performance on Classical LOCO Dataset}
\label{sec:locoresultssota}

- describe trend seen in experiments section\newline
- review classes with generally good performance\newline
- review (nature of) anomalies that were especially easy/hard\newline

- discuss differences in performance among IAD approaches\newline
-> also if some classifiers were better/had more weaknesses in specific regions that others\newline

- compare sPRO of classifiers to sPRO of approaches specifically designed for LOCO like \cite{LOCODentsAndScratchesBergmann2022}\newline

%In this section we review the performance of prior introduced anomaly detection methods. All experiments were performed with the same 
%experimental setup as explained in section (referenz of experimental setup section), the conditions explained in section (referenz von methods section über loco) 
%and on the mvtec LOCO dataset \cite{LOCODentsAndScratchesBergmann2022}. 
%The results of inference on the test set can be seen in table x (tabelle mit ergebnissen). As it can be seen, all models scored a significantly 
%lower result on the MVTecAD LOCO dataset than on the normal MVTecAD one(exemplary scores seen in table xy(table mit normalen mvtec scores)). 
%A lower performance is generally to be expected, since firstly logial anomalies are regarded as a more difficult problem than structual 
%ones and secondly the average SOTA performances as seen in table x(tabelle mit ergebnissen) is already closing in on an AUROC of of 1. 
%(den satz rechts von hier müsste man maybe rausmachen oder umschreiben)Therefore there is not much room for further improvement in similar settings, and a worse performance still aknowledgeable as very good. 
%Yet there is an drop in cross-model average AUROC of approcimately (durchschnitts drop ausrechnen), which is a remarkable(synonym) difference. 
%Most other metrics, namely (metrics names), also declined with an respective average of (respective averages). As explained in section 
%(referenz zu metrics section von background), the sPRO (or rather AU-sPRO) was a score introduced in \cite{LOCODentsAndScratchesBergmann2022} to gain an 
%advanced insight on the quality of segmentations. This means that all approaches who either were published before or did not include this 
%paper in their research likely did not include this metric, which holds true for the approahces used for this experiment. Therefore no comparison 
%in sPRO/AU-sPRO can be shown(vllt einfach spro auch für allte ansätze implementieren?? dann kann ich den satz ändern). Comparing the sPRO 
%scores of the SOTA methods in this experiment with the ones from compared to GCAD \cite{LOCODentsAndScratchesBergmann2022} shows asignificantly 
%(abchecken ob wirklich) worse performance.
%Among the different models, the highest scoring one was PatchCore \cite{patchCore2022}. It scored an average (metrics einfügen) feature embedding based approaches like  
%achieved the highest scoring

%Interpretation of results hier, weiß nicht in welche section das eigentlich muss:


\section{Flat Connector}
\label{sec:flatconnectordiscussion}

- same analysis of metrics as above\newline
- also analyse different anomaly types\newline

- first analyse only with above approaches\newline



\section{Ensemble Performance}
\label{sec:ensemblediscussion}

\subsection{Independent Transformation Block}
\label{subsec:ITBfaildiscussion}

- give possible reasons for failure of this methods\newline
-> explain that applying pca to non ensemble features also resulted in this noise, therefore fault must lie with pca\newline
-> give reasons why pca can fail(non linear dependency, ...)\newline


\subsection{Stacking Ensemble}
\label{subsec:stackingdiscussion}

- interpret similar to loco survey section\newline
- also look at how the two experiments performed(hierarchy vs variety)\newline




